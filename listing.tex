
\renewcommand\thechapter{\CYRA\arabic{chapter}}
\renewcommand\appendixname{Приложение~А}
\chapter*{~~Этап подготовки входных данных}\label{chap:a}
\addcontentsline{toc}{chapter}{\appendixname~~Стадия предобработки}
\renewcommand{\thefigure}{\CYRA\arabic{figure}}

Листинг~\ref{lst:code} содержит реализацию функции предобработки технологической карты.

\begin{lstlisting}[language=Golang,caption={Предобработка технологической карты},label=lst:code]
    func PrepareExperiment(exp Experiment, hooks ...Hook) Experiment {
        exp.Resources = CloneResources(&exp)
        for _, route := range exp.ProductionRoutings {
            for opName, opDesc := range route.Operations {
                for resName, conf := range opDesc.BindTo {
                    res, ok := exp.Resources[resName]
                    if !ok {
                        panic("internal error")
                    }
                    for _, hook := range hooks {
                        opDesc.BindTo[resName] = hook(res, conf)
                    }
                }
                route.Operations[opName] = opDesc
            }
        }
        return exp
    }
\end{lstlisting}

Листинг~\ref{lst:code2} содержит реализацию функции создания шаблона продукта.

\begin{lstlisting}[language=Golang,caption={Создание шаблона продукта},label=lst:code2]
    func mkProductTemplate(pd ProductionRouting, productType ProductTypeName) productTemplate {
        result := productTemplate{
            events:    make([]abstractEvent, 2*len(pd.Operations)),
            resources: make(map[ResourceName][]bindTemplate, len(pd.Operations)),
        }
        type keydefine struct {
            opName    OperationName
            eventType eventType
        }
        dict := make(map[keydefine]index, 2*len(pd.Operations))
        if len(pd.Operations) == 0 {
            panic("No Operations")
        }
        var count int
        keys := make([]OperationName, 0, len(pd.Operations))
        for key := range pd.Operations {
            keys = append(keys, key)
        }
        sort.Sort(byOperationNames(keys))
        for _, opName := range keys {
            dur := pd.Operations[opName].Duration()
            if *dur <= 0 || dur == nil {
                panic("Duration <= 0!")
            }
            e1Ix, e2Ix := index(count*2+0), index(count*2+1)
            count++
            abstractOpID := abstractOperationID{productType, opName}
            result.events[e1Ix] = abstractEvent{abstractOpID, BeginEvent, nil, nil, []index{e2Ix}}
            result.events[e2Ix] = abstractEvent{abstractOpID, EndEvent, &abstractEventDef{e1Ix, *dur}, []index{e1Ix}, nil}
            dict[keydefine{opName, result.events[e1Ix].eventType}] = e1Ix
            dict[keydefine{opName, result.events[e2Ix].eventType}] = e2Ix
            binds, ok := pd.Operations[opName]
            if !ok {
                panic("There is no such resource")
            }
            for resName, conf := range binds.BindTo {
                result.resources[resName] = append(
                    result.resources[resName],
                    bindTemplate{conf, []index{e1Ix, e2Ix}},
                )
            }
        }
    
        for _, seq := range pd.Sequence {
            aI, aOk := dict[keydefine{seq.Next, BeginEvent}]
            bI, bOk := dict[keydefine{seq.Prev, EndEvent}]
            if !aOk || !bOk {
                panic("Error with constraints")
            }
    
            result.events[aI].prev = append(result.events[aI].prev, bI)
            result.events[bI].next = append(result.events[bI].next, aI)
        }
        return result
    }
\end{lstlisting}

Листинг~\ref{lst:code3} содержит реализацию функции создания заказа.

\begin{lstlisting}[language=Golang,caption={Создание заказа},label=lst:code3]
    func generateProductionTasks(templates map[ProductTypeName]productTemplate, order []OrderItem) []productionTask {
        var result []productionTask
        snCounter := map[ProductTypeName]int{}
        for _, item := range order {
            if item.NumberOfProducts == 0 {
                panic("There is a product, but no NumberOfProducts")
            }
            template, ok := templates[item.ProductType]
            if !ok {
                panic("product template is not defined: " + item.ProductType)
            }
            if _, ok := snCounter[item.ProductType]; !ok {
                snCounter[item.ProductType] = 0
            }
            for i := 0; i < item.NumberOfProducts; i++ {
                result = append(
                    result,
                    productionTask{
                        productTypeName: item.ProductType,
                        productSN:       ProductSN(snCounter[item.ProductType]),
                        productTemplate: &template,
                    },
                )
                snCounter[item.ProductType]++
            }
        }
        return result
    }
\end{lstlisting}

\renewcommand\appendixname{Приложение~В}
\chapter*{~~Этап планирования}\label{chap:b}
\addcontentsline{toc}{chapter}{\appendixname~~Этап планирования}

Листинг~\ref{lst:code4} содержит реализацию функции создания задания на од­ну единицу продукции.

\begin{lstlisting}[language=Golang,caption={Создание заданий},label=lst:code4]
    func (plan *ProductionPlan) appendTask(task productionTask) {
        sn := task.productSN
        template := task.productTemplate
        offset := len(plan.Events)
        for _, aEvent := range template.events {
            op := plan.appendOperation(aEvent.abstractOperationID, sn)
            v := ConcreteEvent{op, eventType(aEvent.eventType), nil, nil, nil, nil}
            if aEvent.define != nil {
                v.define = &concreteEventDef{nil, aEvent.define.duration}
            }
            plan.Events = append(plan.Events, &v)
        }
        for i := range template.events {
            if plan.Events[offset+i].define != nil {
                startIx := int(template.events[i].define.start)
                plan.Events[offset+i].define.start = plan.Events[offset+startIx]
            }
            for _, j := range template.events[i].prev {
                plan.Events[offset+i].prev = append(plan.Events[offset+i].prev, plan.Events[offset+int(j)])
            }
            for _, j := range template.events[i].next {
                plan.Events[offset+i].next = append(plan.Events[offset+i].next, plan.Events[offset+int(j)])
            }
        }
        for resName, binds := range template.resources {
            res, ok := plan.resources[resName]
            if !ok {
                panic("resource is not defined: " + resName)
            }
            for _, bind := range binds {
                eventsToBind := make([]*ConcreteEvent, len(bind.indexes))
                for i, ix := range bind.indexes {
                    event := plan.Events[offset+int(ix)]
                    eventsToBind[i] = event
                    if event.EventType == BeginEvent {
                        eventsToBind[i].resources = append(eventsToBind[i].resources, res)
                    }
                }
                res.Bind(bind.conf, eventsToBind...)
            }
        }
    }
\end{lstlisting}


Листинг~\ref{lst:code5} содержит реализацию функции планирования.

\begin{lstlisting}[language=Golang,caption={Планирование},label=lst:code5]
    func defineProductionPlan(plan *ProductionPlan, selector func([]*ConcreteEvent) *ConcreteEvent) *ProductionPlan {
        var err error
        for err == nil {
            err = defineNextEventGroup(plan, selector)
        }
        return plan
    }
    func defineNextEventGroup(plan *ProductionPlan, selector func([]*ConcreteEvent) *ConcreteEvent) error {
	events := possibleNextEvents(plan, false)
    events = filterPossibleEventsByResources(events)
    if len(events) == 0 {
		return fmt.Errorf("сan't find any possibleNextEvents")
	}
    event := selector(events)
	// fmt.Printf("Order:%s \n", event.Operation.Name)
	// fmt.Printf("%v - current\n", event)
	var newValue int64
	for _, v := range event.prev {
		if newValue < *v.Value {
			newValue = *v.Value
		}
	}
	for _, res := range event.resources {
		v, isAllow := res.Constrain(event)
		if !isAllow {
			panic("event is not allow to process!!!")
		}
		if newValue < v {
			newValue = v
		}
	}
	event.SetValue(newValue)
	doneResources(event)

	for _, nextEvent := range event.next {
		if nextEvent.define != nil && nextEvent.define.start == event {
			newValue := nextEvent.define.duration + *event.Value
			nextEvent.SetValue(newValue)
			doneResources(nextEvent)
		} 
	}
	return nil
}


\end{lstlisting}

Листинг~\ref{lst:code6} содержит реализацию случайного перебора во время имитационного моделирования.
\begin{lstlisting}[language=Golang,caption={Случайный перебор},label=lst:code6]
    func MkExperimentWithRandom(exp Experiment, conf Config) []*ProductionPlan {
        var Plans []*ProductionPlan
    
        experimentChanel := make(chan *ProductionPlan, conf.NumOfExp)
    
        for i := 0; i < conf.NumOfExp; i++ {
            go func() {
            copyExp := CopyExperimentWithRandom(&exp,int64(i))
            experimentChanel <- MkExperiment(copyExp)
            }()
        
        }
        for i := 0; i < conf.NumOfExp; i++ {
            Plans = append(Plans,<-experimentChanel)
        }
        log.Println("Done")
        file, err := os.Create("test.txt")
    
        if err != nil{
            panic(err)
        }
        defer file.Close()
    
        value, _ := highScore(Plans)
        if conf.Score != nil {
            return scoreSelector(value, *conf.Score, Plans, file)
        }
        return Plans
    }
\end{lstlisting}



\renewcommand\appendixname{Приложение~С}
\chapter*{~~Тестирвоание}\label{chap:c}
\addcontentsline{toc}{chapter}{\appendixname~~Тестирование}

\begin{lstlisting}[language=Golang,caption={Тестирование случайным перебором в процессе планирования},label=lst:code7]
    func TestMkExperimentWithRandom(t *testing.T) {
        exp := imcore.Experiment{
            Name: "order1",
            ProductionRoutings: map[imcore.ProductTypeName]imcore.ProductionRouting{
                "LokomotivA": {
                    Operations: map[imcore.OperationName]imcore.OperationDesc{
                        "op1": imcore.NewOperationDbg(9, imcore.BindConfs{"mechan": resource.DemandForStaff{Amount: 4}}),
                        "op2": imcore.NewOperationDbg(8, imcore.BindConfs{"mechan": resource.DemandForStaff{Amount: 3}}),
                        "op3": imcore.NewOperationDbg(7, imcore.BindConfs{"mechan": resource.DemandForStaff{Amount: 2}}),
                        "op4": imcore.NewOperationDbg(6, imcore.BindConfs{"mechan": resource.DemandForStaff{Amount: 1}}),
                        "op5": imcore.NewOperationDbg(5, imcore.BindConfs{"mechan": resource.DemandForStaff{Amount: 4}}),
                        "op6": imcore.NewOperationDbg(4, imcore.BindConfs{"mechan": resource.DemandForStaff{Amount: 3}}),
                        "op7": imcore.NewOperationDbg(3, imcore.BindConfs{"mechan": resource.DemandForStaff{Amount: 2}}),
                        "op8": imcore.NewOperationDbg(2, imcore.BindConfs{"mechan": resource.DemandForStaff{Amount: 1}}),
                        "op9": imcore.NewOperationDbg(1, imcore.BindConfs{"mechan": resource.DemandForStaff{Amount: 1}}),
                    },
                    Sequence: []imcore.PrevAndNext{{Prev: "op1", Next: "op2"},
                        {Prev: "op1", Next: "op3"},
                        {Prev: "op1", Next: "op4"},
                        {Prev: "op1", Next: "op5"},
                        {Prev: "op2", Next: "op6"},
                        {Prev: "op3", Next: "op6"},
                        {Prev: "op3", Next: "op7"},
                        {Prev: "op4", Next: "op7"},
                        {Prev: "op4", Next: "op8"},
                        {Prev: "op5", Next: "op8"},
                        {Prev: "op6", Next: "op9"},
                        {Prev: "op7", Next: "op9"},
                        {Prev: "op8", Next: "op9"},
                    },
                },
            },
            OrderItems: []imcore.OrderItem{
                {ProductType: "LokomotivA", NumberOfProducts: 30},
            },
            Resources: map[imcore.ResourceName]imcore.Resource{
                "mechan": &resource.Staff{Total: 4},
                "DEBUG":  imcore.DebugResource{},
            },
            EventSelector: imcore.SimpleEventSelector,
        }
    
        i := 1
        p := &i
        plans := imcore.MkExperimentWithRandom(exp, imcore.Config{NumOfExp: 150, Score: p})
        fmt.Println(plans)
    }

\end{lstlisting}