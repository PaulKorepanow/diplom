\chapter{Глава}
\label{cha:chapter}

\todo[inline, color=red]{так выглядит заметка в тексте}
\section{Новый раздел}

\subsection{Оформление текста}
Пустая строка

- это новый абзац\\%\\ - интервал между строками

\textbf{Жирный}

\textit{Курсив}

\sout{Зачерктунтый}

\textcolor{red}{Красный}\\

Перечисление без нумерации
\begin{itemize}
	\item Первый
	\begin{itemize}
		\item 1-1
		\item 1-2
	\end{itemize}
	\item Второй\\
\end{itemize}

Перечисление с нумерацией

\begin{enumerate}
	\item Первый
	\item Второй\\
\end{enumerate}

Ссылка на литературу\todo[color=green]{Заметка на полях} \cite{Pup09}.

\subsection{Таблица}


\begin{table}[h]
	\caption{\label{tab:canonsummary}Измерительные характеристики цифровой камеры Canon EOS 400D.} %Заголовок
	\begin{center}%по центру
		\begin{tabular}{|c|c|} % два столбца с границами и выравниванием по центру
			\hline % горизонтальная граница
			% Первая строка, "&" - hfpltkbntkm cnjk.wjd
			Параметр 		& 		Значение \\
			\hline
			Разрешение		 & $3888 \times 2592$ \\
			Размер сенсора	 & $22.2 \times 14.8$ мм \\
			АЦП & 12~bit\\
			\hline
			\multicolumn{2}{|c|}{Результаты измерений} \\ % Объединяем столбцы с выравниванием по центру
			\hline
			Темновое смещение (BLO) & 256 \\
			Максимальный линейный сигнал & 3070~DN \\
			Значение насыщения & 3470~DN \\
			\hline
		\end{tabular}
	\end{center}
\end{table}


\subsection{Формулы}
Формула или математическое выражение в тексте $a=b$

Формула или математическое выражение
$$
a=b+c
$$
с отдельной строки.\\

Горизонтальный $b\: a$ пробел

Толстый горизонтальный $c\;e$ пробел

Индексы $a_b^c$

Нумерованные формулы
\begin{eqnarray}
\label{eq:a}%ссылка на формулу
a=b+c
\end{eqnarray}

Так ссылаться на формулу, рисунок или таблицу в тексте (\ref{eq:a})

Формула в несколько строк
\begin{eqnarray*}
\label{new_formula}
&a=b,\\
&c=e
\end{eqnarray*}

$$
\alpha,  \theta, \upsilon, \beta, \vartheta, \pi, \phi, \gamma, \iota, \varpi, \varphi,
, \delta , \kappa , \rho , \chi
, \epsilon , \lambda , \varrho , \psi
, \varepsilon , \mu , \sigma , \omega
, \zeta , \nu , \varsigma
, \eta, \xi , \tau
$$

\subsection{Рисунки}
\begin{figure}[H]
	\center{\includegraphics[width=1\linewidth]{fig/fig.jpeg}}
	\caption{Зависимость сигнала от шума для данных.}
	\label{ris:image}
\end{figure}

Сылка на рисунок \ref{ris:image}.


\begin{lstlisting}[style=pseudocode,caption={Алгоритм оценки дипломных работ}]
def EvaluateDiplomas():
	for each student in Masters:
		student.Mark := 5
	for each student in Engineers:
		if Good(student):
			student.Mark := 5
		else:
			student.Mark := 4
\end{lstlisting}


%%% Local Variables:
%%% mode: latex
%%% TeX-master: "rpz"
%%% End:
